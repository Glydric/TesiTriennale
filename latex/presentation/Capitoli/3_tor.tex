\section{Tor § Onion v2}
\begin{frame}{Tor}
    \begin{wrapfigure}{1}{0.2\textwidth}
        \centering
        \includegraphics[width=0.2\textwidth]{TorLogo}
    \end{wrapfigure}

    La rete Tor è la più famosa implementazione di Onion, grazie all'apporto delle seguenti migliorie:
    \begin{itemize}
        \item Circuiti telescopici
        \item Proxy di applicazione tramite SOCKS
        \item Controllo di congestione
        \item Directory server
        \item Politiche di uscita variabili
        \item Controllo d'integrità end-to-end
    \end{itemize}
\end{frame}

\subsection{Obiettivi}
\begin{frame}{Obiettivi}
    L'obiettivo principale della rete TOR è quello di garantire l'anonimato dell'utente finale, e scoraggiare eventuali attaccanti, sono stati quindi definiti i seguenti obiettivi:
    \begin{itemize}
        \item Usabilità, la rete deve essere utilizzabile da chiunque, questo è un'aspetto fffondamentale per garantire l'anonimato.
        \item Semplicità, la rete deve essere semplice da utilizzare, in modo da non scoraggiare gli utenti meno esperti.
    \end{itemize}
\end{frame}

\subsection{Network Design}
\begin{frame}{Network Design}
    \begin{wrapfigure}{1}{0.2\textwidth}
        \centering
        \includesvg[width=0.3\textwidth]{overlayNetwork}
    \end{wrapfigure}

    La rete Tor è una rete che esiste al di sopra delle esistenti reti
\end{frame}
