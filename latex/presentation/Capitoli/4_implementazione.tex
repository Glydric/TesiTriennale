\section{Implementazione}
\begin{frame}{Implementazione}
    Per implementare un servizio onion è necessario utilizzare il \textbf{Proxy Onion} per inoltrare le richieste dalla rete Tor al server web, il proxy gestisce la generazione delle chiavi, dell'indirizzo e la definizione e connessione con gli \textbf{introduction points}, oltre alla generazione del descriptor e la sua pubblicazione nei Directory Servers. \\
    In particolare la nostra implementazione userà \textbf{Onion V3} (dato che Onion V2 è stato deprecato) con alcune migliorie tra cui la maggior sicurezza degli indirizzi. \\
    Useremo un server \textbf{Linux EC2} di AWS per ospitare il proxy onion e un \textbf{server nginx}.
\end{frame}

\begin{frame}
    Nel sistema è necessario installare il pacchetto nginx (web server) e dopo una serie di configurazioni dei repository possiamo installare tor. 
    \newline
    Configuriamo il file di configurazione torrc con \lstinline{HiddenServiceDir /var/lib/tor/hidden_service} (per indicare la directory dove verranno salvate le chiavi) e \lstinline{HiddenServicePort 80 unix:/var/run/website.sock} (per indicare il tipo e il modo di connessione con il web server).
\end{frame}