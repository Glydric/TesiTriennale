%%%%%%%%%%%%%%%%%%%%%%%%%%%%%%%%%%%%%%%%%%%%%%%%%%%%%%%%%%%%
%%  This Beamer template was created by Matteo Rucco..
%%  Anyone can freely use or modify it for any purpose
%%  without attribution.
%%
%%  Last Modified: December 7, 2014
%%

\documentclass[xcolor=x11names,compress]{beamer}

%% General document %%%%%%%%%%%%%%%%%%%%%%%%%%%%%%%%%%
\usepackage{graphicx}
\usepackage{tikz}
\usetikzlibrary{decorations.fractals}
%%%%%%%%%%%%%%%%%%%%%%%%%%%%%%%%%%%%%%%%%%%%%%%%%%%%%%


%% Beamer Layout %%%%%%%%%%%%%%%%%%%%%%%%%%%%%%%%%%
\useoutertheme[subsection=false,shadow]{miniframes}
\useinnertheme{default}
\usefonttheme{serif}
\usepackage{palatino}
\usepackage{setspace,textcomp,soul}

\setbeamerfont{title like}{shape=\scshape}
\setbeamerfont{frametitle}{shape=\scshape}
\definecolor{bluUnicam}{RGB}{27,43,74}
\definecolor{redUnicam}{RGB}{219,0,36}
\definecolor{orangeUnicam}{RGB}{234,114,40}
\setbeamercolor*{lower separation line head}{bg=redUnicam} 
\setbeamercolor*{normal text}{fg=bluUnicam,bg=white} 
\setbeamercolor*{alerted text}{fg=red} 
\setbeamercolor*{example text}{fg=black} 
\setbeamercolor*{structure}{fg=orangeUnicam} 
\setbeamercolor*{frametitle}{fg=redUnicam}


\setbeamercolor*{palette tertiary}{fg=orangeUnicam,bg=bluUnicam} 
\setbeamercolor*{palette quaternary}{fg=black,bg=black!10} 


\theoremstyle{definition} \newtheorem{esempio}{Esempio}
\theoremstyle{definition}
\newtheorem{definizione}{Definition} \theoremstyle{plain}
\newtheorem{teorema}{Theorem}


%%%%%%%%%%%%%%%%%%%%%%%%%%%%%%%%%%%%%%%%%%%%%%%%%%

\begin{document}


%%%%%%%%%%%%%%%%%%%%%%%%%%%%%%%%%%%%%%%%%%%%%%%%%%%%%%
%%%%%%%%%%%%%%%%%%%%%%%%%%%%%%%%%%%%%%%%%%%%%%%%%%%%%%
\section{\scshape}
\begin{frame}
\title{\color{redUnicam}{TITOLO TESI}}
%\subtitle{SUBTITLE}
\begin{center}
%\author{
%	Matteo Rucco\\
%	{\it University of Camerino - TOPDRIM Collaboration}\\
%	
%	}
	
\author[shortname]{{\textbf{Leonardo Migliorelli}\inst{1}}\\
\fontsize{8pt}{10}\selectfont{nome.cognome@studenti.unicam.it}
\institute[shortinst]{\fontsize{6pt}{6}\selectfont{\inst{1}University of Camerino, Italy}}}
\date{
	\begin{figure}[htpb!]
	
	\centering{
   \includegraphics[width=0.10\textwidth]{immagini/stemma}
 }
%\today
\end{figure}
	
}
\titlepage
\end{center}
\end{frame}

%%%%%%%%%%%%%%%%%%%%%%%%%%%%%%%%%%%%%%%%%%%%%%%%%%%%%%
%%%%%%%%%%%%%%%%%%%%%%%%%%%%%%%%%%%%%%%%%%%%%%%%%%%%%%
%{
%\usebackgroundtemplate{\includegraphics[width=\paperwidth,height=\paperheight]{./immagini/camerino}}
%\begin{frame}{Computer Science @ Unicam}
%
%The Unicam Group - Formal Methods and Analysis of Complex Systems :
%\begin{itemize}
%\item Professor and TOPDRIM coordinator: E. Merelli http://www.cs.unicam.it/merelli/BioShape-lab/
%\item Researcher: L. Tesei
%\item PhD Students: N. Paoletti, M. Taffi, P. Penna and M. Piangerelli
%\item Students (Bachelor and Master): J. Binchi (joint work with ISI), J. De Berardinis (TDA comparison)
%\end{itemize}
%\end{frame}
%
%}
%%%%%%%%%%%%%%%%%%%%%%%%%%%%%%%%%%%%%%%%%%%%%%%%%%%%%%
%%%%%%%%%%%%%%%%%%%%%%%%%%%%%%%%%%%%%%%%%%%%%%%%%%%%%%
\begin{frame}{Outline}
\tableofcontents
\end{frame}
%%%%%%%%%%%%%%%%%%%%%%%%%%%%%%%%%%%%%%%%%%%%%%%%%%%%%%
%%%%%%%%%%%%%%%%%%%%%%%%%%%%%%%%%%%%%%%%%%%%%%%%%%%%%%
\section{Introduction}
\begin{frame}{I frame}


\begin{minipage}[c]{0.65\textwidth}
The goal of this project is ....
\end{minipage}
\begin{minipage}[c]{0.25\textwidth}
%\includegraphics[width=1.6\textwidth]{./immagini/wp}
\end{minipage}



\end{frame}
%%%%%%%%%%%%%%%%%%%%%%%%%%%%%%%%%%%%%%%%%%%%%%%%%%%%%%
%%%%%%%%%%%%%%%%%%%%%%%%%%%%%%%%%%%%%%%%%%%%%%%%%%%%%%
\subsection{Complex Networks, Toplogical Data Analysis and Biology}
\begin{frame}{Complex Networks}
Complex networks are one of the most used tool for studying complex systems, weighted networks are becoming a more and more important tool to detect either the presence and the intensity of relations among groups of nodes in a network.

\centering{
%\includegraphics[width=0.50\textwidth]{immagini/complex}
}

\end{frame}
%%%%%%%%%%%%%%%%%%%%%%%%%%%%%%%%%%%%%%%%%%%%%%%%%%%%%%
%%%%%%%%%%%%%%%%%%%%%%%%%%%%%%%%%%%%%%%%%%%%%%%%%%%%%%


\section*{Acknowledgements}
\begin{frame}{Acknowledgements}
\begin{center}
\includegraphics[width=6.5cm]{./immagini/thanks}
\end{center}
\end{frame}

%%%%%%%%%%%%%%%%%%%%%%%%%%%%%%%%%%%%%%%%%%%%%%%%%%%%%%
%%%%%%%%%%%%%%%%%%%%%%%%%%%%%%%%%%%%%%%%%%%%%%%%%%%%%%

\end{document}
