\chapter{Conclusioni}

La tesi evidenzia come la rete Onion abbia un ruolo significativo nell'ambito della privacy e dell'anonimato online. 
Dalle sue primissime versioni fino alla più recente rete Onion, che è stata analizzata in questo lavoro, ha dimostrato di essere un mezzo efficace ed efficiente per proteggere la comunicazione e la navigazione degli utenti in un mondo in cui la privacy viene sempre più trattata come vera e propria moneta.
D'altro canto l'uso delle reti Onion ha sollevato problemi legali e sociali riguardo all'identificazione di attività illegali sul web, dimostrando la necessità di definire un equilibrio tra privacy e prevenzione dei crimini.
\\

È stata mostrata anche la facilità con cui è possibile creare un servizio Onion configurando il proxy Tor per reindirizzare il traffico dalla rete Onion verso un web server. 
Implementando anche elementi interessanti come la creazione di un dominio personalizzato e la gestione dei pagamenti.
\\

L'argomento in assoluto più interessante durante la ricerca, stesura e studio della tesi è stato proprio l'analisi delle tecnologie sviluppate per consentire agli utenti di godere di una maggiore sicurezza online, assieme all'associazione del dominio alla coppia di chiavi asimmetriche necessarie per garantire l'anonimato. 

\section{Sviluppi futuri}
Nella scrittura del lavoro di tesi sono emersi 3 possibili sviluppi futuri:
\begin{itemize}
    \item \textbf{Analisi di OnionShare}, è un software che permette di condividere file, inviare messaggi e creare un servizio onion in modo anonimo e sicuro sfruttando la rete TOR. Oltre a come sia possibile implementare un onion proxy all'interno di un'applicazione desktop per fornire un servizio in maniera anonima e sicura.
    \item \textbf{Analisi di alternative} come \emph{I2P} e \emph{Freenet}, come differiscono da TOR e come le diverse tecnologie implementate influenzano la sicurezza e l'anonimato degli utenti.
    \item Analisi di come la rete TOR venga usata dai cittadini di paesi dove vi è una forte censura e sorveglianza sulle attività svolte online, come delle implicazioni legali che gli utenti TOR potrebbero dover affrontare. 
\end{itemize}
