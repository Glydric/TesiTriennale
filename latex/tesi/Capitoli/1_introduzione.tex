\chapter{Introduzione}
\label{chap:intro}

Le moderne tecnologie di rete consentono una rapida comunicazione da ogni parte del mondo, avvicinando culture altrimenti distanti migliaia di chilometri. Due dei più grandi temi del nostro secolo sono la privacy e l'anonimato, parlando di reti internet ogni connessione tra client e server passa per una moltitudine di router che conoscono esattamente l'indirizzo (e quindi l'identità) del mittente e del destinatario. Con un po' di conoscenze non è complicato scoprire questi dati e sfruttarli a proprio vantaggio, le stesse big company spesso usano l'indirizzo IP con cui ci si connette al loro sito per tracciare l'utente e fornirgli articoli e pubblicità mirata o vendere i medesimi dati a terzi. 
La Rete Onion è stata creata per risolvere esattamente questo problema

% \section{Motivazione}
\section{Obiettivi}

La tesi ha come principale obiettivo la dimostrazione di come è possibile generare un servizio/web server che sfrutta la rete tor per rendere le connessioni anonime e rendere anonimo lo stesso server, creare un hostname Tor personalizzato, far si che il servizio sia raggiungibile attraverso un motore di ricerca onion e fornire un sistema di pagamento integrato per i prodotti del servizio

\section{Struttura della Tesi}
Nel primo capitolo viene introdotto l'onion routing, i possibili utilizzi e le mix networks. 
Il secondo capitolo introduce la rete Tor con le relative migliorie apportate all'onion routing e gli obiettivi prefissati per l'implementazione della rete.
Il terzo capitolo introduce ai servizi onion, alla rete V3 che ha apportato alcune modifiche importanti per la sicurezza, precondizioni necessarie per l'implementazione che verrà mostrata dopo uno studio delle tecnologie disponibili
