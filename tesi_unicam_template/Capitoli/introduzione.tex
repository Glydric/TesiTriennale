\chapter{Introduzione}
\label{chap:intro}

Le moderne tecnologie di rete consentono una rapida comunicazione da ogni parte del mondo, avvicinando culture altrimenti distanti migliaia di chilometri. Due dei più grandi temi del nostro secolo sono la privacy e l'anonimato, in particolare parlando di reti internet ogni connessione tra client e server passa per una moltitudine di router che conoscono esattamente l'indirizzo (e quindi l'identità) del mittente e del destinatario. Con un pò di conoscenze non è complicato scoprire questi dati e sfruttarli a proprio vantaggio, le stesse big company spesso usano l'indirizzo IP con cui ci si connette al loro sito per tracciare l'utente e fornirgli articoli e pubblicità mirata o vendere i medesimi dati a terzi. 
La Rete Onion è stata creata per risolvere esattamente questo problema, implementando le giuste tecnologie per proteggere gli utenti dall'analisi del traffico e dalle intercettazioni


\section{Motivazione}
\section{Obiettivi}

La tesi ha come principale obiettivo la specifica della rete onion, dalle mix networks alla prima versione di Onion fino a Onion v3 e la relativa implementazione di un servizio completo

\section{Struttura della Tesi}
