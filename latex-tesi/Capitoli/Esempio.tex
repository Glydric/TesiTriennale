\chapter{Capitolo Esempio}
\label{chap:CapitoloEsempio}

\section{Sezione Esempio}
\label{sec:real-time}

Esempio elenco puntato ...
\begin{itemize}
	\item item 1
	\item item 2
	\item item 3
\end{itemize}
Esempio elenco numerato ...
\begin{enumerate}
	\item item 1
	\item item 2
	\item item 3
\end{enumerate}

\subsection{Subsection Esempio}
\label{sec:es}

\begin{lstlisting}[caption={Esempio di listing}, style=javaScriptCode]
	GET /chat HTTP/1.1
	Host: server.example.com
	Upgrade: websocket
	Connection: Upgrade
	Sec-WebSocket-Key: dGhlIHNhbXBsZSBub25jZQ==
	Origin: http://example.com
	Sec-WebSocket-Protocol: chat, superchat
	Sec-WebSocket-Version: 13
\end{lstlisting} 

\footnote{\url{https://github.com/Glydric/TesiTriennale}} (Esempio di link in footnote).

\newpage
nuova pagina

\begin{table}[htbp]
	\begin{center}
		\begin{tabular}{|l|l|l|l|l|l|}
			\hline
			Versione & Chrome & Firefox & Internet Explorer & Opera & Safari \\
			\hline
			76 & 6 & 4.0 & No & 11.00(disabilitato) & 5.0.1\\
			\hline
			7 & No & 6.0 & No & No & No \\
			\hline
			10 & 14 & 7.0 & HTML5 Labs & ? & ?\\
			\hline
			RFC 6455 & 16 & 11.0 & 10 & 12.10 & 6.0\\
			\hline
		\end{tabular}
	\end{center}
	\caption{Esempio di Tabella}
	\label{tab:browser}
\end{table}

% il riferimento può essere fatto con qualsiasi \label
riferimento tabella \ref{tab:browser}

