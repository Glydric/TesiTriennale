\newpage
\chapter{Tor § Onion v2}
\label{chap:Capitolo2}
Nel 2002 viene presentata la rete Tor, diventata open source 2 anni dopo è l'implementazione più conosciuta di onion routing \\
Tra le varie migliore abbiamo 
\begin{itemize}
    \item La segretezza del canale, nella versione originale un nodo poteva forzare altri onion router a decriptare un traffico precedentemente registrato. La rete TOR invece sfrutta una tecnica di circuiti telescopici, le chiavi generate dal client in un determinato circuito sono di breve durata, non è quindi possibile utilizzarle successivamente per decriptare il vecchio traffico
    \item L'implementazione del proxy di applicazione attraverso lo standard SOCKS, consente alla maggior parte del traffico TCP di funzionare senza modifiche. Precedentemente era necessario implementare un proxy per ogni applicazione, e di conseguenza generare un circuito per ogni applicazione, con conseguente duplicazione di chiavi. Il proxy implementato in TOR invece genera il circuito a livello di TCP e più processi applicativi possono utilizzarlo. Per garantire la non tracciabilità di un utente nello stesso stream dati usato da più applicazioni è stato implementato il meccanismo dei \emph{Rotating Circuits} che genera ogni minuto un nuovo circuito se quello precedente non sta essendo usato
    
    \item Controllo di congestione, un sistema decentralizzato che sfrutta ack end-to-end per garantire l'anonimato\ref{sec:CongestionControl}
    \item Directory Server, nodi più fidati di altri che descrivono le informazioni di rete in maniera sicura e affidabile
    \item Politiche di uscita variabili, ogni nodo possiede delle politiche che specificano le connessioni consentite e rifiutate, fondamentale in una rete distribuita fatta da volontari
    \item Controllo di integrità end-to-end, viene eseguito un controllo di integrità nel momento in cui il pacchetto esce dalla rete per garantire che i contenuti non sono stati alterati
\end{itemize}
Tor però non è completamente sicura, infatti non filtra informazioni di privacy nel corpo del messaggio come invece avviene in altri sistemi come Privoxy o Anonymizer e non offre garanzie in caso di attacco end-to-end che concerne sia sorgente che destinazione
\cite{onionv2}

\section{Obiettivi}
L'obiettivo principale della rete TOR è la creazione di una rete che possa rendere gli attacchi molto più complessi da portare a termine scoraggiando cosi un possibile hacker, da questo principio cardine sono derivati gli altri obiettivi, tra cui
\begin{itemize}
    \item Usabilità, la rete Tor a differenza degli altri sistemi che implementano le Mix-Networks\ref{sec:OnionMixNetwork} predilige la bassa latenza\footnote{Il che rende la rete adatta all'utilizzo tramite un web browser} e l'usabilità, un aspetto fondamentale se si vuole garantire l'anonimato\footnote{Maggiori sono i nodi più semplicemente si può garantire l'anonimato}.
    Da questo derivano altri obiettivi come
    \begin{itemize}
        \item Basse latenze, determinate anche dal fatto che più applicazioni possono usare lo stesso circuito TCP senza doverne generare uno nuovo per ogni stream dati, questo consente di ridurre il delay causato dalla criptografia asimmetrica
        \item Essere implementabile con meno configurazioni possibili, per incrementare il numero di servizi che possano attirare utenti
        \item Essere multi piattaforma, per incrementare la base di utenti
    \end{itemize}
    \item Semplicità, la rete deve essere facile da comprendere, doveva essere usabile nel mondo reale e non doveva essere troppo costosa
\end{itemize}

\cite{ChaumMixes}
\section{Network Design}
A differenza di altri sistemi che usano una rete peer-to-peer, Tor è una \emph{overlay network}\footnote{Una rete virtuale creata sfruttando una rete fisica preesistente}, questa scelta è derivata dalle possibili vulnerabilità di una rete basata sugli utenti, infatti un attaccante potrebbe compromettere il traffico leggendo o manipolando i dati. \\
Ogni onion router è rappresentato come un processo software che mantiene due tipi di chiavi
\begin{itemize}
    \item Chiave d'identità, una chiave di lunga durata usata per firmare i pacchetti garantendo agli altri nodi della rete l'autenticità del messaggio e delle informazioni contenute, tra cui indirizzo, bandwidth, exit policy, ecc..
    \item Chiave onion, una chiave di breve durata usata per decriptare le richieste all'interno dei circuiti utente
\end{itemize}

Un elemento chiave della rete TOR sono le celle, pacchetti di dimensione fissa a 512 bytes, come ogni tipo di pacchetto sono divisi in header e payload, l'header contiene l'identificativo del circuito e un comando che indica come gestire il payload. Il comando può essere
\begin{itemize}
    \item \textbf{Padding} per mantenere viva la connessione
    \item \textbf{Create} per creare un nuovo circuito
    \item \textbf{Destroy} per eliminare un circuito
\end{itemize}
Inoltre il tipo di comando definisce il tipo della cella
\begin{itemize}
    \item \textbf{Control}, pacchetti di controllo gestiti dal primo router che li riceve, per questo non vengono mai inoltrati
    \item \textbf{Relay}, trasporta stream dati per cui ha necessità di un header aggiuntivo contenente l'identificativo streamID, il checksum per il controllo di integrità, la dimensione del payload e un comando di relay. Il comando di relay può essere 
    \begin{itemize}
        \item \textbf{Begin} per iniziare uno stream
        \item \textbf{End} per terminare uno stream
        \item \textbf{Teardown} per terminare uno stream in modo forzato, usato per quelli "rotti"
        \item \textbf{Connected} per rispondere al relay begin dell'OP, informandolo che lo stream è stato creato con successo
        \item \textbf{Extend} per estendere un circuito di un router 
        \item \textbf{Truncate} per eseguire un teardown di una parte del circuito
        \item \textbf{Sendme} usato per il controllo di congestione
        \item \textbf{Drop} 
    \end{itemize}
\end{itemize}

L'utente per generare il circuito segue un processo di negoziazione incrementale:
\begin{enumerate}
    \item Come primo passo l'utente, o meglio l'Onion Proxy crea e invia una richiesta \emph{relay} al primo nodo nel percorso scelto
    \item Avviene la condivisione della chiave simmetrica tramite l'handshake di Diffie-Hellman, la creazione dell'ID del circuito e di conseguenza la creazione della connessione con il primo nodo (R1)
    \item L'utente invia una richiesta relay extend indicando a R1 l'indirizzo del secondo nodo nel percorso scelto (R2), R1 inizia una connessione con R2 definendo un nuovo id e associando la connessione OP-R1 con la connessione R1-R2, OP e R2 non si conoscono a vicenda e comunicano solo tramite l'intermediario R1. In fine R1 invia all'utente le informazioni del nuovo nodo tra cui la chiave simmetrica per il relativo strato
    \item Questa operazione viene effettuata, richiedendo all'ultimo router corrente di creare una connessione con il suo successivo, finché il circuito non viene completato\cite{ChaumMixes}
\end{enumerate}
Ogni applicazione, in base alle proprie necessità, invia le richieste di stream TCP all'Onion Proxy, che sceglie il circuito più recente (oppure genera un nuovo circuito) e sceglie un exit node, solitamente l'ultimo router. 
A questo punto l'OP invia una cella \emph{relay begin} con un identificativo casuale all'exit node, la risposta dell'exit node conferma l'esistenza del nuovo stream TCP e l'OP può accettare i dati delle applicazioni TCP ed inoltrarli nella rete Onion \\

\subsection{Controllo di Congestione} \label{sec:CongestionControl}
Questo design ha qualche problema derivato dal fatto che più OP potrebbero scegliere lo stesso percorso OR-OR, generando un bottleneck e saturando la rete. 
Le normali tecnologie di controllo di congestione non possono funzionare in una rete del genere, i pacchetti devono rispettare il proprio percorso e non possono cambiarlo, questo ne renderebbe impossibile la lettura. Per implementare un sistema di controllo di congestione sono stati sviluppati 2 meccanismi
\begin{itemize}
    \item \textbf{Controllo di Circuito}, ogni OP possiede le informazioni di due finestre per ogni OR, mentre ogni OR possiede le informazioni delle finestre dell'OP per ogni circuito di cui fanno parte. Le 2 finestre iniziano da un valore di 1000 e decrementano a ogni cella, esse sono
    \begin{itemize}
        \item \textbf{Packaging window}, tiene traccia del numero di celle che un OR può inviare verso l'OP, quando un router riceve abbastanza celle dati (100) invia un relay sendme all relativo OP con $streamID = 0$, quando un OR riceve un relay sendme incrementa la finestra di packaging (di 100) per il relativo OP, se la finestra di packaging raggiunge 0 il nodo che sia un Onion Router o un'Onion Proxy smette di ricevere ed inviare celle dal circuito attendendo il sendme
        \item \textbf{Delivery window}, tiene traccia del numero di celle che un OR è in grado di inviare fuori dalla rete
    \end{itemize}
    \item \textbf{Controllo di Stream}, simile al controllo di circuito ma il meccanismo è applicato ad ogni stream TCP separatamente. Ogni stream inizia con packaging = 500 con incrementi di 50 ad ogni relay sendme. Il relay sendme, a differenza del controllo a livello di circuito, viene inviato quando il numero di bytes che devono essere inviati è inferiore al threshold di 10 celle
\end{itemize}

\cite{onionv2}
\newpage
\section{Directory Servers} \label{sec:DirectoryServers}
I directory server sono un piccolo sottogruppo di onion router utilizzati per tracciare i cambiamenti nella topologia di rete. 
In particolare un directory server agisce come un server HTTP accessibile dai client per ottenere lo stato della rete e la lista dei router aggiornata periodicamente dagli stessi OR. \\
Quando il directory server riceve un aggiornamento da un OR prima di tutto controlla la chiave d'identità del router, garantendo che un attaccante non possa fingersi un OR manomettendo la rete \cite{ChaumMixes}. 
Il software di accesso alla rete onion è precaricato con le informazioni sui directory server e le relative chiavi, le informazioni di rete vengono aggiornate periodicamente dall'OP. \\

I directory server sono anche fondamentali per la connessione agli onion services, infatti ogni servizio creato genera un \emph{Onion Service Descriptor} contenente una lista degli introduction points e la chiave pubblica, il pacchetto viene criptato con la chiave privata e inviato al directory server che quindi agisce come fosse un DNS server per gli onion services. 
Quando l'utente tenta la connessione a un sito onion richiede e riceve dal directory server il relativo \emph{Descriptor} per l'indirizzo onion, usa la chiave pubblica, derivata dalla stringa dell'indirizzo onion a cui vuole connettersi, per decriptare il pacchetto \\
Nel caso il Directory Server fosse compromesso e contenesse un Descriptor malevolo generato per reindirizzare il traffico\footnote{Che però non potrà avere la stessa chiave privata}, l'indirizzo onion che possediamo non sarebbe in grado\footnote{tramite la chiave pubblica al suo interno nascosta} di decriptare le informazioni, dato che sono state criptate con una differente chiave privata
\cite{OnionServicesOverview}.
Non c'è neanche la necessità di determinare se le informazioni sono correte, perché a patto che la chiave privata non sia resa pubblica, nessuno potrà manomettere il Descriptor senza renderlo illeggibile e/o \emph{"indecifrabile"}

\section{Tor Browser}
Una delle principali vulnerabilità della rete Tor è la possibilità che un servizio web crei un codice JavaScript\footnote{che viene eseguito sul browser dell'utente} malevolo in grado di ottenere informazioni sull'indirizzo dell'utente deanonimizzandolo. 
Gli sviluppatori di Tor Browser hanno inserito appositamente un sistema di sicurezza che blocca gli script, questo però di contro porta molti siti a non funzionare correttamente. \\
Essendo la rete Tor fortemente basata sulla criptografia la navigazione di un sito onion tramite Tor non fa comparire alcun avviso di sicurezza in caso di mancanza di HTTPS, dato che il traffico è già criptato. 
Da tenere inoltre in considerazione che un certificato proveniente da una Certificate Authority potrebbe generare problemi di anonimizzazione del servizio a causa della Certificate Transparency\cite{CertificateTransparency} \\
% TODO parla dei tor relay in maniera più specifica

% \section{Vulnerabilità} %Qui si parla delle vulnerabilità di onion v2 per cui si è passati a v3