\chapter{Onion Routing}
\label{chap:Capitolo1}

\begin{figure}[htpb!]
    \centering
    \includesvg[width=\textwidth]{SpiegazioneOnion.svg}
    \caption{Vita di un pacchetto onion}
    \label{fig:routing}
\end{figure}

La \textbf{rete onion} è una rete distribuita composta dall'insieme di \textbf{router onion} che agiscono come nodi di rete e collaborano per portare un pacchetto dalla sorgente alla destinazione. Il tutto avviene senza che nessun nodo possa conoscere contemporaneamente l'host sorgente e l'host di destinazione, grazie alla criptografia a strati del pacchetto, per cui ogni strato viene criptato con una chiave differente e può essere decriptato solo dal nodo con la stessa chiave simmetrica, scoprendo cosi le informazioni sul prossimo nodo. Il nodo finale (exit node) può infine decriptare l'intero messaggio e scoprirne il corpo. Questo meccanismo è derivato dallo studio di David Chaum riguardo alle mix networks. La risposta del pacchetto segue lo stesso criterio sfruttando nodi, algoritmi e chiavi differenti. Questo concede all'utilizzatore di rimanere completamente anonimo, cosa che non può essere avvenire con la semplice criptografia SSL che agisce esclusivamente sul corpo del messaggio \\
Come prima operazione per utilizzare la rete onion il client deve generare un nuovo circuito definendo il percorso di nodi che ogni pacchetto dovrà seguire, iniziando dal primo nodo vengono scambiate informazioni crittografiche come l'algoritmo e le chiavi da usare, poi si usano le informazioni finora ottenute per ottenere quelle del prossimo nodo e cosi via fino a che non si è generato tutto il circuito, grazie a questo meccanismo neanche durante la generazione del circuito è possibile risalire al client. Viene usata la crittografia asimmetrica per scambiare le chiavi simmetriche tra il client e ogni router, questo viene fatto in quanto la crittografia asimmetrica è molto costosa e viene quindi usata solo alla generazione del circuito, successivamente gli strati vengono criptati e decriptati con la stessa chiave simmetrica. Questo consente alla rete onion ad avere una bassa latenza che è incrementata solo dal numero di onion router nel percorso e non dalla tecnologia che non è distante da quella usata in HTTPS 
\cite{OnionRouting}

\newpage
\section{Applicazioni di utilizzo}
L'onion routing può essere usato con una moltitudine di protocolli e applicazioni, tra i più comuni troviamo HTTP(S), FTP, SSH, SMTP, DNS e VPNs. L'utilizzo di molti dei protocolli più comuni avviene tramite gli onion proxies, i quali sono suddivisi in tre proxy layer logici
\begin{itemize}
    \item Un proxy che genera e gestisce le connessioni, per operare ha necessità di conoscere la topologia e i percorsi verso altri nodi, tutte le informazioni vengono distribuite in modo sicuro all'interno della rete a ogni nuovo nodo che si connette 
    \item Un proxy chiamato \emph{\textbf{“Application Specific Proxy”}}, a una connessione la relativa applicazione invia al proxy il pacchetto che normalmente invierebbe al server di destinazione, il proxy si occupa di convertire lo stream di dati in un formato accettato dalla rete onion
    \item Un proxy opzionale chiamato \emph{\textbf{“Application Specific Privacy Filter”}} che sanifica lo stream di dati rimuovendo informazioni che potrebbero identificare la sorgente
\end{itemize}

\cite{OnionRouting} \\
I proxy possono essere configurati in molteplici modi, tra i quali vi è la possibilità di eseguire il proxy in un server remoto e sfruttare la rete Tor senza dover installare il software in ogni dispositivo, che quindi non ne deve gestire la computazione

\newpage
\section{Onion come Mix Network}
La rete Tor è una delle molteplici reti basate sullo studio di David Chaum sulle Mix network, il suo studio si concentrò sulla rete basata sulla crittografia asimmetrica e sui sistemi di risposta senza conoscere l'identità del destinatario. Tutta la rete doveva funzionare senza necessità di un'ente centrale a gestire le connessioni. \\
In particolare abbiamo chiave pubblica K e chiave privata P, abbiamo inoltre
\begin{itemize}
    \item K(x) la funzione che cripta x con la chiave pubblica, può solo essere decriptato con la chiave P
    \item P(x) la funzione che cripta x con la chiave privata, può solo essere decriptato con la chiave K
\end{itemize}
Abbiamo quindi $P(K(x)) = K(P(x)) = x$ \\
Con questa tecnica però qualcuno potrebbe determinare il contenuto di un messaggio creando una copia identica in quanto $y = x → K(y) = K(x)$, per risolvere questo problema si esegue la crittografia del messaggio inserendo una stringa casuale R ottenendo quindi chiavi sempre diverse tramite le funzioni $K(x,R)$ e $P(x,R)$, questa tecnica è chiamata sealing \\
In una rete questo sistema viene implementato in maniera ridondante, il messaggio M viene criptato insieme a una stringa casuale R con la chiave pubblica del destinatario Kb, il tutto viene criptato assieme all'indirizzo B e a una stringa casuale R1 con la chiave pubblica K1 del nodo 1, questo processo potrebbe essere esteso con N nodi (mix) rendendo quindi la rete molto sicura \ 
$K1(R1, Kb(R0, M), B)$ \\
Quando il nodo 1 riceve il pacchetto lo decripta con la sua chiave privata scartando R1 ottiene $Ka(R0, M), B$. Inoltra quindi il nuovo pacchetto a B \\
Il destinatario di un pacchetto deve avere la possibilità di rispondere senza conoscere l'indirizzo di A. \\
A sfrutta il proprio indirizzo reale per generare un indirizzo non tracciabile, in particolare genera due chiavi pubbliche Ka e R1, cripta il proprio indirizzo A assieme alla chiave R1 come stringa casuale usando la chiave del mix \\
$K1(R1, A), Ka$ \\
Quando B deve rispondere al pacchetto usa Ka per criptare il messaggio e $K1(R1, A)$ come indirizzo di destinazione, il mix M1 che riceve il pacchetto usa la chiave privata P1 per decriptare l'indirizzo di destinazione (A) ed usa la stringa casuale R1 come ulteriore chiave per criptare il messaggio \\
$K1( R1, A ), Ka( R0, M ) => A, R1( Ka( R0, M ) )$ \\
Con questo sistema B può rispondere ad A, non conoscendo il vero indirizzo di A, il mix non conosce il contenuto del messaggio ma conosce l'indirizzo di destinazione e A è l'unico che può decriptare il contenuto del messaggio \\
Da considerare che M1 è il primo mix nel percorso da A a B, mentre tra M1 e B possono esserci N nodi ed il messaggio può quindi essere criptato più volte nel percorso da B a M1 \\
M1 viene quindi considerato un nodo di uscita dalla rete dato che è l'unico che conosce il vero indirizzo di destinazione 

\cite{ChaumMixes}