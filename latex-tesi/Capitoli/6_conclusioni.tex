\chapter{Conclusioni}

La tesi evidenzia come la rete Onion abbia un ruolo significativo nell'ambito della privacy e dell'anonimato online. 
Dalle sue primissime versioni fino alla più recente rete Onion, che è stata analizzata in questo lavoro, ha dimostrato di essere un mezzo efficace ed efficiente per proteggere la comunicazione e la navigazione degli utenti in un mondo in cui la privacy viene sempre più trattata come vera e propria moneta.
D'altro canto l'uso delle reti Onion ha sollevato problemi legali e sociali riguardo all'identificazione di attività illegali sul web, dimostrando la necessità di definire un equilibrio tra privacy e prevenzione dei crimini.
\\

È stata mostrata anche la facilità con cui è possibile creare un servizio Onion configurando il proxy Tor per reindirizzare il traffico dalla rete Onion verso un web server. 
Implementando anche elementi interessanti come la creazione di un dominio personalizzato e la gestione dei pagamenti.
\\

L'argomento in assoluto più interessante durante la ricerca, stesura e studio della tesi è stato proprio l'analisi delle tecnologie sviluppate per consentire agli utenti di godere di una maggiore sicurezza online, assieme all'associazione del dominio alla coppia di chiavi asimmetriche necessarie per garantire l'anonimato. 


